\documentclass{article}
\usepackage[utf8]{inputenc}

\title{FOAR705 - Scoping Exercise}
\author{Jan Jugueta - 44828020}
\date{August 2019}

\begin{document}

\maketitle

\section{Research Area}

Because of the interdisciplinary nature of my academic training in the department of International Studies, my research approach will naturally also be interdisciplinary. Tentatively, I am looking to research the influence that Soviet-style communism had on the development of footballers during Honecker's administration of the German Democratic Republic (East Germany). Frameworks and theories from Sociology, Anthropology and History will all be beneficial when attempting to gain an insight into how footballers were developed, and what impact this had on their performances in the international sporting arena.

\section{Jobs}

\begin{itemize}
  \item I will need to compile a list of primary and secondary sources that will provide both context and insight into the world of youth football development in Germany between 1971 and 1989.
  \item I will then find the libraries and archives that will have these type of sources available.
  \item I will then need to copy and catalogue these sources somewhere (most likely on my computer)
  \item I need to figure out a way to easily find which source I need at any time
  \item Whilst reading all these sources, I will eventually need to delimit my research and focus on a specific question
  \item Write, write, write.
  \item Read, read, read.
  \item Re-write, read and re-write some more.
  \item Have regular meetings with my supervisor.
\end{itemize}


\section{Pains}

One of the biggest challenges that I face whilst undergoing this Masters of Research is that I am at the best of times not very good with time management, and am acutely aware that there really isn't much time to complete the body of work required for submission. This time 'pain' is further exacerbated by the fact that the focus of my research is located in Germany. I will have to find sometime (most likely in late 2019) to go to Germany and collect sources.

Another pain is that I am a serial procrastinator. It seems that I work better under pressure, closer to the due date of an assignment. In the weeks (or months) prior to the due date, I tend to over-contemplate what I'm doing and therefore stall from making any progress.

Another potential pain is that much of the sources I will be dealing with will be in German. I have completed learning German to a competent level of fluency, however, academic and/or bureaucratic use of the German language could prove to be more challenging.

\section{Pain Relievers}

I have begun to use an organizer and a to-do list to help me effectively manage my time. Occasionally I will opt to use public transport and intentional choose longer, slower services as it forces me to use that time to catch up on my reading. A to-do list on my computer and on my phone regularly reminds me of the immediate tasks at hand, along with longer term jobs. Following on advice from another FOAR course, I have been writing regularly (almost daily), which I believe has helped me immensely with my writing fluency.

I have recently discovered an online archive for one of East Germany's more prominent newspapers. With this in mind, I would like a tool that is able to search through the contents of all newspapers published between 1971 and 1989 for particular keywords and either store or tag the articles in someway that correspond to those keywords.

To tackle the pain of having to deal with mostly German texts, I would like a tool that is able to create the English translation of text either under every line, or under every paragraph, almost like the subtitles that you see for TV or film. I am not sure how feasible that is, but it is a suggestion. Failing that, I know that I can fall back on my own German skills, but I feel that it would be take up more of my time.

\section{Gains}

I would like to be more proficient when using technology with my academic work. Although this is the first time that I am using LaTeX, I can see the benefit of using technology like this when compiling a 20,000+ word thesis.

I would like to be a better writer, as it is one of the primary ways that an academic communicates with their peers.

I would like to be a more organised researcher. I currently have quite an ad-hoc approach when it comes to synthesising ideas.

\section{Gain Creators}

Where allowable, I will from this point on use LaTeX with all my academic work. My belief is that I will get more comfortable with LaTeX the more I use it. Ideally, by the time I need to write my MRes thesis, and potentially a PhD in the future, writing in LaTeX would be no different than writing in Microsoft Word.
I am also interested to see what I will learn in regards to DataCarpentry. I hope to use what I learn in whatever research I undertake.

I will also continue the practice of writing daily. Writing is just like any other skill, which improves the more one does it. I have found that the practice of writing daily has vastly reduced the idle time that I spent sitting infront of a keyboard not knowing what to type. I have also recognised an increase in writing fluency.

I would like to investigate more research strategies. To date, I have not experimented much with mind maps or research journals. I have heard from colleagues that such strategies have been quite beneficial in their academic work. I will also continue to regularly talk with my academic colleague, regardless of their disciplinary background. From experience, regular communication with fellow researchers has often given me out of the box ideas that have helped propel my work forward when it has stalled. I also feel that talking about the emotional aspect of research can help alleviate the sometimes lonely feeling that is often associated with research.

\end{document}
